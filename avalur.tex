INTRODUCTION


Newton’s  Second  Law  of  Motion  describes  the relationship between force, mass, and acceleration [1]: 


                             (1)


where  is the force in newtons (N),  is the mass in grams (g), and  is the acceleration in m/s2. This law implies that for a given force, an object’s acceleration is inversely proportional to its mass. As mass increases, the acceleration decreases [1]:   


                                                        (2)


The significance of  lies in its ability to predict how objects will accelerate when subjected to different forces. For this experiment, we use a one-dimensional (1D) cart system where a constant force is exerted by a 100g weight. By varying the mass of the cart and measuring the time it takes to travel a set distance, we can calculate the acceleration and examine the relationship between mass and acceleration. 
We seek to verify Newton’s Second Law and hypothesize that as the mass of the cart increases, the acceleration will decrease, consistent with Equation (2). To test this hypothesis, we conducted multiple trials using a one-dimensional cart system. Known masses were placed on the cart, and a constant force was applied via a 100g weight. By comparing the accelerations for different masses, we examined the relationship between mass and acceleration to verify Newton’s Second Law [1].


  

FIG. 1. Materials Used in Lab Setup.
Display of the materials used in the lab experiment: a cart (to carry varying masses), a near-frictionless track (to reduce resistance), a pulley system (to apply a constant force), and masses. These are the necessary materials and components for investigating the relationship between mass, force, and acceleration.
  

FIG. 2. Experimental Setup.
A free-body diagram illustrating the experimental setup. The cart (m1​) is connected to a hanging mass (m2) through a string, which is hung over a pulley, with forces labeled to represent the tension (T), gravitational force (𝑚𝑔), and normal force (𝑁) acting on the system.


METHODS AND MATERIALS


Cart Acceleration Test


Acceleration tests ( = 18) were conducted using a one-dimensional cart system along a fixed 80 cm track. The experimental setup included a wheeled cart with a base mass of 500 g and a near-frictionless track, both from PASCO Scientific [2], to ensure consistent performance with minimal resistance for accurate measurements. Additional masses of 20g, 50g, 100g, 200g, 500g, and 1,000g were used to vary the cart’s total mass. A 100g weight was suspended to apply a constant force to the cart. The mass was given on the bottom of each individual weight. 
The cart was released from a designated starting point 80 cm from the endpoint, and the time taken to travel the distance was recorded using a stopwatch with 0.01-second precision. Each mass configuration was tested in three separate trials ( = 3)  to account for measurement variability. For each setup, the average time and corresponding standard deviation were calculated from the three trials to summarize the timing data, presented as mean ± one standard deviation.
By averaging the times, we obtained a general measure of the time taken by the cart to travel the set distance, reducing the impact of inconsistencies in individual measurements. In addition to calculating the average time, each trial was treated as an independent measurement to calculate individual acceleration values for each mass. This approach allowed us to observe the variability across trials and provided a more thorough analysis of the acceleration for each configuration.
To calculate the acceleration, we used the formula [1]:  


                             (4)


where 𝑎 is the acceleration,  is the distance traveled (80 cm or 0.8 m), and  is the time taken for the cart to travel that distance. By substituting the calculated average times for each mass, we computed the corresponding average acceleration values. Additionally, we calculated acceleration from each trial independently, which enabled us to analyze the standard deviation and consistency of the acceleration measurements for each mass configuration.
To assess the variability of time measurements across trials, we calculated the standard deviation for each mass configuration. The standard deviation provides insight into the consistency of the time values and helps quantify any fluctuations due to measurement inaccuracies. This calculation allowed us to observe the spread of time values for each mass, providing a measure of the reliability of our results. The standard deviation of the acceleration was also calculated for each mass, which allowed us to evaluate the consistency of acceleration measurements across trials.


RESULTS


FIG. 3 presents the time measurements for a cart with varying masses, recorded over three trials. For each mass, the times for Trial 1, Trial 2, and Trial 3 are displayed, along with the calculated average time and the standard deviation. 


  

FIG. 3. Time Measurements for Different Masses, with Trial Times, Average Time (s), and Standard Deviation.


FIG. 4 displays the calculated acceleration for a cart with varying masses, derived independently from each of the three trials. For each mass, the acceleration values from Trial 1, Trial 2, and Trial 3 are shown, alongside the acceleration calculated from the average time, accompanied by the standard deviation. 


  

FIG. 4. Calculated Acceleration From Each Trial and Average Acceleration with Standard Deviation for Cart with Varying Masses.


FIG. 5 presents the relationship between acceleration and mass for the cart system under the constant applied force. The plot includes individual data points representing the calculated acceleration for each trial at each mass, without using the averaged accelerations. These data points allow for a detailed view of the variability in acceleration across trials for each mass. Additionally, the theoretical predicted curve, calculated based on the formula  is shown, where  is the 100g (or 0.1 kg) mass providing a constant force,  is the gravitational acceleration (approximately 9.81 m/s2), and  is the varying mass converted to kg [1]. This was depicted prior in FIG. 2.




  

FIG. 5. Graph of Acceleration vs. Mass: Individual Trial Data Points and Predicted Curve Under Constant Force


DISCUSSION


Is Newton’s Second Law Verified?


As observed in FIG. 3, increasing the mass on top of the cart generally resulted in an increase in the time taken to travel the set distance of 80 cm. For example, with 20g, the time recorded across three trials ranged from 0.95 to 1.10 seconds. When the largest mass (1,000g) was added, the time increased, ranging from 1.56 to 1.90 seconds. These individual trial results provide a reliable primary basis for analyzing the effect of mass on time and, subsequently, on acceleration. The trial data clearly show a trend of increasing time with added mass, consistent with Newton’s Second Law [1].
The calculated accelerations, shown in FIG. 4, further reinforce this relationship. By analyzing the individual acceleration values across the three trials for each mass, a clear inverse relationship between mass and acceleration emerges. For instance, with 20g, the acceleration values across trials ranged from approximately 1.32 m/s2 to 1.77 m/s2. As the mass increased to 1,000g, the acceleration values dropped significantly, ranging from approximately 0.44  m/s2 to 0.66  m/s2 across trials. This inverse trend across individual measurements strongly supports Newton’s Second Law, where a constant force applied to an increasing mass yields lower acceleration [1]. 
FIG. 5 further corroborates this trend by plotting individual acceleration values for each trial against the theoretical predicted curve. The individual data points closely follow the expected inverse relationship–for all three trials, as mass increases, acceleration decreases–although some slight deviations from the predicted curve are observed. These minor discrepancies likely result from experimental errors such as slight variations in the release of the cart or timing precision, which will be discussed later. Despite these small deviations, the consistent downward trend in acceleration as mass increases validates the predicted inverse relationship and strongly aligns with Newton’s Second Law [1].
While the averages and standard deviations shown in FIG. 3 and FIG. 4 provide a general summary of the results, they serve as secondary evidence to the results of the experiment. For example, the average acceleration, which was calculated from the average time, for 20g was 1.51 m/s2 with a standard deviation of 0.23 m/s2 , and for 1,000g it was 0.57 m/s2 with a standard deviation of 0.12 m/s2. These averages are useful for observing the general trend, but are less detailed and therefore less reliable than the individual trial values, which provide a clearer view of the consistency and variability within each mass configuration. The standard deviations, however, add context by showing the variability within trials, which remains relatively low, indicating that individual trial data is consistent and dependable [3].
To conclude, our findings–primarily supported by individual trial data and secondarily further corroborated by averaged values–demonstrate a consistent inverse relationship between mass and acceleration under constant force. This strong, inverse trend, even in the presence of minor experimental deviations, provides compelling support for Newton’s Second Law, illustrating that as mass increases, acceleration decreases proportionally [1]. The individual trial data serve as the most reliable evidence, while averages offer a general confirmation, reinforcing the reliability of our results.


Sources of Experimental Error


While the track used in this experiment was near-frictionless, it is essential to acknowledge that some friction is unavoidable. The near-frictionless plane was chosen to minimize the effects of friction on the acceleration measurements, as a lot of friction can introduce significant experimental error by opposing the motion of the cart. Despite this, tiny variations in friction could still have influenced the results.
Timing inaccuracies likely introduced error due to the manual use of a stopwatch, especially at higher masses where precise measurement was required over longer intervals [5]. To improve accuracy, we could use an automated timing system, such as photogates, which would eliminate human reaction time errors and provide precise start and stop measurements [4]. This change would ensure that timing measurements are consistent and highly accurate across trials. 
Additionally, slight inconsistencies in the cart’s release, such as variations in initial positioning or angle, may have affected the measurements. To fix this, we could use a mechanical release mechanism to standardize the release process [4]. Such a mechanism would ensure that the cart starts from the exact same position and orientation in each trial, minimizing variability due to manual handling. This adjustment would help control for any small discrepancies caused by differences in the release method, leading to more reliable acceleration data.






BRIEF ACKNOWLEDGMENTS


We would like to thank our AP Physics C Mechanics teacher, Dr. Evangelista, for his guidance and support throughout this experiment, as well as for providing our group with all the necessary materials from his classroom. His assistance helped us better understand the principles underlying Newton’s Second Law of Motion and how to apply them in a practical lab setting. We would also like to thank our peer reviewers for helping us to revise our lab and providing valuable, constructive feedback. 


Individual Contributions


For this experiment, each member of our group contributed to various parts of the experiment to ensure a thorough and collaborative process. Dia timed each trial during the experiment and assisted in writing the lab report. Jophy recorded the data for the experiment, performed the data calculations, created the figures, and also assisted in writing the report. Srilekha prepared the experimental setup and managed the string during each trial. Anika tested each weight individually and released the cart in each trial. 


REFERENCES


[1] Knight, R. D. (2017). Physics for Scientists and Engineers: A   ….Strategic Approach (4th ed.). Pearson.


[2].PASCO Scientific. (n.d.). PASCO Scientific - Physics …..Equipment & Science Lab Supplies. Retrieved from ….https://www.pasco.com


[3].Starnes, D. S., Tabor, J., Yates, D., & Moore, D. S. (2015). ….The Practice of Statistics (5th ed.). W.H. Freeman and ….Company.
\
[4].Taylor, J. R. (1997). An Introduction to Error Analysis: The …..Study of Uncertainties in Physical Measurements (2nd ed.). ….University Science Books.


[5] Hetzler, R. K., Stickley, C. D., Lundquist, K. M., & Kimura, I. …..F. (2008). Reliability and accuracy of handheld stopwatches …..compared with electronic timing in measuring sprint …..performance. Journal of Strength and Conditioning Research, ….22(6), 1969-1976.  https://doi.org/10.1519/JSC.0b013e318185
….f36c